\documentclass[jou]{apa6}

\usepackage[utf8]{inputenc}
\usepackage[english]{babel}
 
\usepackage{hyperref}
\usepackage{apacite} 


%\title{Reproducibility and Reusability}
%\title{Replication and Reuse in Computational Modelling}
\title{Beyond reproducibility}

%\shorttitle{}

\twoauthors{Olivia Guest}{Nicolas P. Rougier}

\twoaffiliations{Department of Experimental Psychology\\University of Oxford, United Kingdom}{INRIA Bordeaux Sud Ouest\\Institute of Neurodegenerative Diseases, France}

%olivia.guest@psy.ox.ac.uk
%nicolas.rougier@inria.fr
\abstract{}

\begin{document}

\maketitle

Computational modelling is the process by which phenomena found in complex
systems are expressed algorithmically and/or mathematically to be eventually
simulated. The creation of such simulations is useful because it allows us to
test whether our understanding is sophisticated enough to create credible
working models of the phenomena we are studying. In neuroscience and cognitive
science especially, computational modelling comprises more than just capturing
a single phenomenon, it also provides an implementation of the theory. Thus, it
gives scientists a method of allowing their ideas to be executed, i.e., for
emergent properties to appear when they are implemented and run as code
\cite{mcclelland09}.

\subsection*{Reproducibility is good}

For such models to be evaluated by peers, it is important to ensure these
models are reproducible, that is, that they offer the possibility of being
recreated based on their sole specification: the details deemed important
enough to be included within the article \cite{hinsen15}. This should be
possible, ideally, without needing to contact the authors for advice, and
critically, without even looking at the original code \cite{cooper14}. If
there is enough information in the specification to successfully recreate
the codebase from scratch, then the model is said reproducible, adding
more credence to both the model itself and its overarching theoretical
framework. If not, then even if the experiments can be carried out successfully
within the original codebase, then the model is not able to be reimplemented
(given the current specification). This is a cornerstone of the scientific
method. If a paradigm or experiment is resistant to reproduction, related
theories are also brought into question. An important corollary is that if
failures to reproduce happen often so-called reproducibility crises occur —
wherein important theories lose their credibility in a domino-like effect.


\subsection*{Reproducibility is a first step}

In 1665, the first academic journal ("Le Journal des sçavans") was published in
Paris and since then (350 years ago), nothing has really changed: research
communication is still and mostly a matter of text, figures, tables and
references (with the major difference that we don't print them anymore). But
where does the code fit in such framework ? Footnotes ? Compressed archive in
the supplementary material ? A link to a repository somewhere on the internet ?
Even if more and more journals are correctly requesting code, as well as raw
data, it is nonetheless striking to see that in the meantime they do not
provide the infrastructure for hosting such code or data, as if these were
second-class citizen. On the other side of the internet however, things are
moving and changing fast. Computer science has progressed a lot in the last few
years and now offers a large number of innovative tools (notebook, containers,
dynamic figures, etc.) that facilitate the share and the re-use of
models. Incidently, the first article with an interactive figure has just been
published a few months ago while the IPython notebook for the gravitational
wave discover has been made public.

\subsection*{Towards Science 2.0}

Maybe the time is ripe for the community to envisage radical changes in the
way of sharing and reusing computational research. There is already some new
initiative such as the ReScience\footnote{\url{rescience.github.io}} journal
that encourage the replication of models and lives on github.


\bibliographystyle{apacite}

\bibliography{ref}

% The following space works around a bug in typesetting the references, where the hanging indent of the last reference is incorrectly set.
\hspace*{1cm}


\end{document}
\documentclass[jou]{apa6}

\usepackage[utf8]{inputenc}
\usepackage[english]{babel}
 
\usepackage{hyperref}
\usepackage{xcolor}
\hypersetup{
    colorlinks,
    linkcolor={red!50!black},
    citecolor={blue!50!black},
    urlcolor={blue!80!black}
}
\usepackage{apacite} 

\title{Reproducibility in Computational Research}
\shorttitle{Reproducibility in Computational Research}

\twoauthors{Olivia Guest}{Nicolas P. Rougier}

\twoaffiliations{Department of Experimental Psychology\\University of Oxford, United Kingdom}{INRIA Bordeaux Sud-Ouest, Talence, France\\
Institut des Maladies Neurodégénératives, Université Bordeaux, Centre National de la Recherche Scientifique, UMR 5293, Bordeaux, France\\
LaBRI, Université de Bordeaux, Institut Polytechnique de Bordeaux, Centre National de la Recherche Scientifique, UMR 5800, Talence, France}

%olivia.guest@psy.ox.ac.uk
%nicolas.rougier@inria.fr
\abstract{}

\begin{document}

\maketitle

Reproduction is a cornerstone of the scientific method.
If a paradigm or experiment is resistant to attempts at reproduction, related theories are also brought into question.
An important corollary is that if failures to reproduce happen often then so-called reproducibility ``crises'' occur --- wherein theories lose their credibility in a domino-like effect.

\subsection*{What is computational reproducibility?}

Computational modelling is the process by which phenomena found in complex systems are expressed algorithmically and/or mathematically.
The creation of such simulations is useful because it allows us to test whether our understanding is sophisticated enough to create credible working models of the phenomena we are studying.
In neuroscience and cognitive science especially, computational modelling comprises more than just capturing a single phenomenon, it also  implements a theory.
It gives scientists a method of allowing their ideas to be executed, i.e., for emergent properties to appear when they are implemented and run \cite{mcclelland09}.

For models to be evaluated, tested, criticized or even rejected, it is important to first ensure they are reproducible \cite{topalidou15}.  
That is, that they can be recreated based on their specification --- the details deemed important enough to be included in the accompanying article \cite{hinsen15}.
Ideally, this should be possible without contacting the authors for advice, and critically, without looking at the original code \cite{cooper14}.
If the specification is sufficient to successfully recreate the codebase from scratch, then the model is said to be reproducible.
This adds further credence to both the model and its overarching theoretical framework.
If not, and the model cannot be recreated, then even if the experiments can be carried out successfully within the original codebase, the model is not reproducible \cite{crook13}.


\subsection*{How do we share computational research?}

There have been few substantial changes within scholarly communication and research dissemination since 1665, when the first academic journals (\textit{Le Journal des Sçavans} and \textit{Philosophical Transactions of the Royal Society}) were published.
Dissemination of scientific discoveries via publishers continues to consist primarily of static text and figures.
However, most research is underpinned by, if not wholly comprised of, code, which is inherently dynamic.
Given code forms the backbone of modern scientific research, it is perhaps unusual that its position within this framework is not clear.
For example, it is not straightforward where codebases should be placed: in a footnote (with code assured to be available upon request), in supplementary materials, or in an online repository?
Even though more journals are requesting code from us, as well as raw data, few publisher-backed repositories exist.
It is striking that an overwhelming number of journals make no provisions for and offer little guidance on hosting these files or indeed facilitating access to them.


\subsection*{Is time ripe for changes?}

However, the open source and open science communities proposed solutions to some of these problems without publishers' aid nor mediation.
A large number of innovative software tools (e.g., the binder project) make modelling work accessible.
Some researchers have taken matters into their own hands and created resources for best practice \cite<e.g., version control:>{blischak16, eglen16, wilson16}.
While others lead by example, e.g., \citeA{ogrean16} published an article with an interactive figure, and the LIGO Open Science Center released extensive amounts of data and code \cite{ligo16}.
In the same vein, the \emph{ReScience journal} encourages the reproduction of modelling work.

Is the scientific community ready to rise to the challenge and drive changes with respect to: associating articles with original codebases in a transparent way and, more broadly, making sure computational theories are specified and implemented in coherent ways?



\bibliographystyle{apacite}

\bibliography{ref}

% The following space works around a bug in typesetting the references, where the hanging indent of the last reference is incorrectly set.
\hspace*{1cm}


\end{document}